\documentclass[13pt, a4paper, twoside]{article}
\usepackage[utf8]{inputenc}
\usepackage{geometry}
\usepackage[czech]{babel}
\usepackage{enumitem}
\usepackage{fancyhdr}
\usepackage{amsmath}
\usepackage{mathtools}
\usepackage{float}
\usepackage{setspace}
\usepackage{multicol}
\usepackage{graphicx}
\geometry{legalpaper, margin=1.05in}
\pagestyle{fancy}
\lhead{\Large Discussion based assessment}
\rhead{\large Matěj Červenka 31.1.2021}
\begin{document}
\begin{enumerate}
\large \onehalfspacing
\item According to MVT $f'(c)$ must equals to the slope of the secant line that
connects $f(0)$ and $f(3)$, so that $c \in (0, 3)$.
First we have to find what is the average rate of the function over the given interval.
    \begin{align*}
        m_{avg} = \frac{f(3)-f(0)}{3-0} = \frac{0}{3} = 0
    \end{align*}
Now we need to find $f'(x)$.
    \begin{align*}
        f'(x) = 3x^2 - 6x
    \end{align*}
Solve for $f'(c)=0$.
    \begin{align*}
        3c^2 - 6c &= 0 /:c \:(\text{x=0 is not on our interval})\\
        c = 2
    \end{align*}
\item It tells us that the function f has absolute maximum and minimum on $<-2,2>$

\item  We must find $f'(x)$ and then solver for $f'(x)=0$. Those points where the 
derivative is 0 are points where the function \emph{usually} changes from decreasing
to increasing and vice versa.
\begin{align*}
    f'(x) = 8(2x+1)^3\\
    0 = 8(2x+1)^3 \Rightarrow x=-0.5
\end{align*}
Now the domain of $f$ was split into $x<-0.5$ and $x>-0.5$. Now we take number
from each interval and plug it in $f'(x)$ to see whether it is positive or negative.

$x<-0.5 \to f'(x)>0 (increasing)$

$x>-0.5 \to f'(x)>0 (increasing)$\\
The function is increasing on its domain (all real numbers).
\item The global minimum of this quadratic function on $<-4,2>$ will be there
where $f'(x)=0$ and $f''(x)>0$.
\begin{align*}
    f'(x)=2x + 2 \Rightarrow x=-1
\end{align*}
We don't need to find the second derivative because the quadratic polynomial member
is positive. We know for sure that the mimimum will be at $x=-1$. We just need
to solve for $f(-1)$.
\begin{align*}
    f(-1) = -6
\end{align*}
The global minimum is $-6$.
\item To determine the concavity of graph of the function $f$ we need to be looking at
at $f''$. The second derivative of $f''(x)=e^x$. We know that exponential functions are
always larger than zero $f''(x)>0$. The function $f$ will be concave up on the given interval.

\item The function has an inflex point at $x=0$ because $f''(0)=0$. And the graph is
concave down at $x=-1$ and concave up at $x=1$. If $x=0$ is the only point of inflection
of the function $f$ then the graph of the function $f$ is concave down on $(-\infty,0)$ and concave up 
on  $(0, \infty)$.

The graph of the function $f$ has one inflection point at $x=3$. The function is 
concave up on $(-6,3)$ and concave down on $(3, 4)$. $f(-3)$ is relative minimum because $f''(-3)>0$.

\item 

\end{enumerate}
\end{document}
