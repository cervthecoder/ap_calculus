\documentclass[13pt, a4paper, twoside]{article}
\usepackage[utf8]{inputenc}
\usepackage{geometry}
\usepackage[czech]{babel}
\usepackage{enumitem}
\usepackage{fancyhdr}
\usepackage{amsmath}
\usepackage{mathtools}
\usepackage{float}
\usepackage{setspace}
\usepackage{multicol}
\usepackage{graphicx}
\geometry{legalpaper, margin=1.05in}
\pagestyle{fancy}
\lhead{\Large Discussion based assessment}
\rhead{\large Matěj Červenka 1.2.2022}
\begin{document}
\begin{enumerate}
\large \onehalfspacing
\item \textbf{A.} The points of inflection are at $x=2, x=4, x=6$ because the tangent
slope of the tangent lines at these points are 0 $\to$ $f''(x)=0$. That means
that the graph of $f$ changes its concavity at these points.


\textbf{B.} The function $f$ has has relative maximum at $x=7$ because there
is a critical point $f'(7)=0$ and $f''(7)<0$.


\textbf{C.} The graph is concave up on $0<x<2$ and $4<x<6$ because $f''(x)>0$
for those intervals.

\item \textbf{A.} To identify the relative extrema of $g$ we need to find the critical
points. We know that $f'(x)=0 or DNE$ at those points.

\begin{align*}
	f'(x) = 3x^2 - 12\\
	\text{Now we need to set it equal to zero} \\
	x = \pm 2 
\end{align*}

We know that the critical point of $g$ is at $x=\pm 2$. Now we need to look
at the function $g''$ to determine wether it is maximum, minimum or neither of those.

\begin{align*}
	g''(x) = 6x
\end{align*}

If we plug the x-coordinates of the critical points into the second derivative
we'll know that $g''(-2)<0$ and $g''(2)>0$. We now know that there's maximum
at $x=-2$ where $g(-2)=16$ and there's a minimum at $x=2$ where $g(2)=-16$. 


\textbf{B.} To determine the concavity of the function we need to find
the inflection points ($g''(x)=0$) which will split the function into concave
up and concave down areas.

\begin{align*}
	0 &= 6x \\
	x &=0
\end{align*}

The point of inflection of $g$ is at $x=0$. From the previous part we know
that for $x<0$ $g''(x)<0$ and for $x>0$ $g''(x)>0$. The function is 
concave down for $x<0$ and concave up for $x>0$.

\item \textbf{A.} To find the relative we need to find critical points
of graph of the function $f$ where there's also a sign change in $f'(x)$.
There's only one extrema on this interval at $x=4$ because $f''(4)=0$ and
for $x<4$ $f'(x)<0$ and for $x>4$ $f'(x)>0$, this also means that there's 
a relative minumum at $x=4 \to f(4)=-4$


\textbf{B.} To find whether the statement is true we'll need to apply the 
MVT for the function $f'(x)$. That means that the slope of the secant line 
between points $(2, f'(2))$ $(4, f(4))$ is zero.

\begin{align*}
	m = \frac{f'(4)-f'(2)}{4-2} = \frac{0}{2} = 0
\end{align*}

This means that there has to be at least on point on $f'$ on $(2, 4)$ where
$f''(c)=0$.




\end{enumerate}
\end{document}









