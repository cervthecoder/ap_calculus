\documentclass[13pt, a4paper, twoside]{article}
\usepackage[utf8]{inputenc}
\usepackage{geometry}
\usepackage[czech]{babel}
\usepackage{enumitem}
\usepackage{fancyhdr}
\usepackage{amsmath}
\usepackage{mathtools}
\usepackage{float}
\usepackage{setspace}
\usepackage{multicol}
\geometry{legalpaper, margin=1.05in}
\pagestyle{fancy}
\lhead{\Large Discussion based assessment}
\rhead{\large Matěj Červenka 16.10.2021}
\begin{document}
\begin{enumerate}
    \large \onehalfspacing
    \item
    \[
        f(x)=
        \begin{dcases}
            4x-7 & x \leq 2 \\
            e^{x-2} & x > 2
        \end{dcases}
    \]
    To prove that the function $f$ is continous at $x=2$ we have
    to look at $\lim_{x\to 2}f(x)$.
    \begin{align*}
        \text{Left-handed limit: }&\lim_{x\to 2^-}f(x) = \lim_{x\to 2}4x-2 = 1\\
        \text{Right handed limit: }&\lim_{x\to 2^+}f(x) = \lim_{x\to 2^+}e^{x-2}  = 1\\
        \text{Therefore: } &\lim_{x\to 2}f(x)=1
    \end{align*}
    In next step of solving this problem we have to show that
    $f(2)$ exists which it does because $f(2)=4\cdot 2 - 7=1$.
    \\The last requirement to prove that $f$ is continous at $x=2$ is
    that $f(2)=\lim_{x\to 2}f(x)$ which it does $\to$ $f(2)=\lim_{x\to 2}f(x)=1$.
    
    \item According to the graph $\Rightarrow$
    \begin{align*}
        \text{Part A: } &\lim_{x\to 4^-}f(x) = 5\\
        \text{Part B: } &\lim_{x\to 4^+}f(x) = 2\\
        \text{Part C: } &\lim_{x\to 2}f(x) = 6
    \end{align*}

    \item Let's (not) find those limits $\Rightarrow$\\
    \textbf{Part A}: We will use our knowledge of simple trigonometric limits
    therefore we know that $\lim_{x\to 0}\frac{sin(nx)}{nx}=1$.
    \begin{align*}
        \lim_{x\to 0} \frac{sin(2x)}{3x} =
         \lim_{x\to 0}\frac{\frac{2}{3}}{\frac{2}{3}} \cdot
         \frac{sin(2x)}{3x} = \frac{2}{3}\cdot \lim_{x\to0}\frac{sin(2x)}{2x}
         =\frac{2}{3} \cdot 1 = \frac{2}{3}
    \end{align*}
    \textbf{Part B}: We have to rationalize and simplify the expression
    to find the limit.
    \begin{align*}
        \lim_{x\to 4}\frac{\sqrt{x-3}-1}{x-4}\cdot
        \frac{\sqrt{x-3}+1}{\sqrt{x-3}+1} =
        \lim_{x\to 4} \frac{x-4}{(x-4)(\sqrt{x-3}+1)} = 
        \lim_{x\to4} \frac{1}{\sqrt{x-3}+1} = \frac{1}{2}
    \end{align*}
    \item Using limit theorems we know that $\Rightarrow$
    \begin{align*}
        &\lim_{x\to 2}[f(x)\cdot g(x)] = \lim_{x\to 2}f(x) \cdot
        \lim_{x\to 2}g(x) \Rightarrow \text{Now we just need to find the values in the graphs}\\
        &\lim_{x\to 2} f(x) = DNE \wedge  \lim_{x\to 2}g(x) = DNE \Rightarrow DNE\cdot DNE = DNE \: \:
        (\text{DNE times DNE is still DNE :)})
    \end{align*}
    \item We consider this piecewise function $h(x)$.
    \[
    h(x)=
    \begin{dcases}
        \frac{x-3}{x^2-9}& x\neq -3; 3\\
        k & x=3
    \end{dcases}    
    \]
    If we want to make $h(x)$ continuous at $(-3, \infty)$ by finding $k$ we have to look
    at the limit as x approaches 3 of $\frac{x-3}{x^2-9}$.
    \begin{align*}
        \lim_{x\to 3}\frac{x-3}{x^2-9} = \lim_{x\to 3} 
        \frac{x-3}{(x-3)(x+3)} = \lim_{x\to3} \frac{1}{x+3}= \frac{1}{6}  
    \end{align*}
    To prevent a removable discontinuity at $x=3$, then
    $h(3) = \lim_{x\to3}f(x) = k = \frac{1}{6}$ $\Rightarrow$
    $k$ must be equal to $\frac{1}{6}$.
\end{enumerate}
\end{document}