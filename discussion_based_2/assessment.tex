\documentclass[13pt, a4paper, twoside]{article}
\usepackage[utf8]{inputenc}
\usepackage{geometry}
\usepackage[czech]{babel}
\usepackage{enumitem}
\usepackage{fancyhdr}
\usepackage{amsmath}
\usepackage{mathtools}
\usepackage{float}
\usepackage{setspace}
\usepackage{multicol}
\geometry{legalpaper, margin=1.05in}
\pagestyle{fancy}
\lhead{\Large Discussion based assessment}
\rhead{\large Matěj Červenka 4.11.2021}
\begin{document}
\begin{enumerate}
    \large \onehalfspacing
    \item a) To find the average rate of change over the interval $[-1, a]$ we will use
    the average rate formula $m=\frac{\Delta y}{\Delta x}$
    \begin{align*}
        &m = \frac{\Delta y}{\Delta x} = \frac{f(a)-f(-1)}{a-(-1)} =
        \frac{\sqrt{a^2-1} - \sqrt{(-1)^2-1}}{a+1} = \frac{\sqrt{a^2-1}}{a+1} =
        \frac{(a+1)^{0.5}(a-1)^{0.5}}{a+1} =\\ &(a+1)^{-0.5} (a-1)^{0.5} = \frac{\sqrt{a-1}}{\sqrt{a+1}}
    \end{align*}
    To find the exact rate of change at $x=-1$ we have to find the left handed limit
    as a approaches -1 because the function $f$ does not exist on the interval $[-1, 1] \to$ function
    is not differentiable on that interval.
    \begin{align*}
        \lim_{a \to -1^-} \frac{\sqrt{a-1}}{\sqrt{a+1}} = -\infty\: \: \text{Therefore $y'$ at $x=-1$ doesn't exist}.
    \end{align*} 
    b) To find the instanatenous change of rate at $x=2$ we will use the limit def. of derivative.
    \begin{align*}
        &\lim_{x \to 2} \frac{f(x)-f(2)}{x-2} = \lim_{x \to 2} \frac{\sqrt{x^2-1}-\sqrt{3}}{x-2}
        = \lim_{x \to 2} \frac{\sqrt{x^2-1}-\sqrt{3}}{x-2} \frac{\sqrt{x^2-1}+\sqrt{3}}{\sqrt{x^2-1}+\sqrt{3}}\\
        &\lim_{x \to 2} \frac{x+2}{\sqrt{x^2-1}+\sqrt{3}} = \frac{4}{2\sqrt{3}} = \frac{2}{\sqrt{3}}
    \end{align*}

    \item a) To estimate the values of derivatives at points we will use the average slope
    formula $m=\frac{\Delta y}{\Delta x}$.\\
    For $x-=2$ we will use points $[-2, g(-2)]\: and \: [-1, g(-1)]$\\
    $g'(2)\approx\frac{3-2}{-2 - (-1)}\approx -1$\\
    For $x=0$ we can say that $g'(0)=0$ because it's the turning point.\\
    For $x=1$ we will use points $[1, g(1)]\: and\: [0, g(0)]$\\
    $g'(1) \approx  \frac{3-0}{0-1} \approx -3$
    
    b) To find the equation of tangent line at $x=1$ we will use the point-slope formula because
    we know the average slope at that point.
    \begin{align*}
        y = -3x +3
    \end{align*}

    \item Let's find the derivatives.\\
    \begin{align*}
        a) \: \: f'(0) + 3h'(0) + 0 = -3\\
        b) \: \: (\frac{f(x)h(x)}{g(x)}-2x^{-1})' = -\frac{2}{9}
    \end{align*}
    \item We can rewrite the function as $f(x)=sin(x)\frac{1}{sin(x)} - \frac{1}{tan(x)}= 1-\frac{1}{tan(x)}=1-cot(x)$
    \begin{align*}
        f'(x) = csc^2(x) \to f'(\pi/4)= 2
    \end{align*}

    \item We can see that the limit we have to evaluate is actually the derivative at $x=1$
    written as definition of derivative using limit. To find the limit we just have to find $f'(1)$.

    \begin{align*}
        f('x) = -4^x\cdot ln(4) \cdot ln(x) - \frac{4^x}{x} \to f'(1) = 0 - 4 = -4
    \end{align*}
    After substituting to the derivative we found $f'(1)=-4$.
\end{enumerate}
\end{document}