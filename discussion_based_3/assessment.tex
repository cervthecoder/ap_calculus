\documentclass[13pt, a4paper, twoside]{article}
\usepackage[utf8]{inputenc}
\usepackage{geometry}
\usepackage[czech]{babel}
\usepackage{enumitem}
\usepackage{fancyhdr}
\usepackage{amsmath}
\usepackage{mathtools}
\usepackage{float}
\usepackage{setspace}
\usepackage{multicol}
\geometry{legalpaper, margin=1.05in}
\pagestyle{fancy}
\lhead{\Large Discussion based assessment}
\rhead{\large Matěj Červenka 24.11.2021}
\begin{document}
\large \onehalfspacing
\begin{enumerate}
    \item We'll use the chain rule to find the derivatives \\ 
    \textbf{A)} $h'(x) = f'(g(x)) \cdot g'(x)$\\
    \textbf{B)} $h'(x) = 3(f(g(x)))^2 \cdot f'(g(x)) \cdot g'(x)$\\
    \textbf{C)} $h'(x)=f'(g(5x)^4) \cdot 4g(5x)^3 \cdot g'(5x) \cdot 5$\\
    \textbf{D)} $h'(x)=f'(arcsin(6x))\cdot \frac{6}{\sqrt{1-36x^2}}$

    \item \textbf{A)} To find $\frac{dy}{dx}$ we need to use the implicit
    differentitation.\\
    \begin{align*}
        &\frac{d}{dx}(x^3+y^3-2xy)= \frac{d}{dx}(2)\\
        &3x^2 + 3y^2\frac{dy}{dx} -2y - 2x\frac{dy}{dx} = 0\\
        &\frac{dy}{dx} =\frac{2y-3x^2}{3y^2-2x}
    \end{align*}
    \textbf{B)} Now we just need to find the slope at (-1, 1) and substitute
    into the point-slope formula.\\
    \begin{align*}
        &\frac{dy|_{y=1}}{dx|_{x=-1}}=\frac{2-3}{3+2} = -\frac{1}{5}\\
        &\text{Tangent line} \Rightarrow y-1 = -\frac{1}{5}(x+1)
    \end{align*}
    \item \textbf{A)} To find f'(x) we'll once again use the chain rule
    and inverse trigonometric differentitation.
    \begin{align*}
        &f'(x) = -\frac{sin(arcsin(x))}{\sqrt{1-x^2}} = -\frac{x}{\sqrt{1-x^2}}\\
        &\text{sin(arcsin(x)) becomes x because if we put the inverse function of x into the function we get x}
    \end{align*}
    \text{B)} Now we'll find the slope at x=0.5 and substitute into the point-slope formula.
    (And also find f(0.5) which is $\sqrt{3}/2$)
    \begin{align*}
        &m = f'(\frac{1}{2}) = -\frac{\frac{1}{2}}{\sqrt{\frac{3}{4}}} = -\frac{1}{\sqrt{3}}\\
        &\text{Tangent line} \Rightarrow y-\frac{\sqrt{3}}{2} = -\frac{1}{\sqrt{3}}(x-\frac{1}{2})
    \end{align*}
    \item \textbf{A)} To find $\frac{dy}{dx}$ we'll just simply use the implicit differentitation.\\
    \begin{align*}
       &\frac{d}{dx}(x^2+y^2-4x -2y + 1 = 0)\\
       &2x + 2y\frac{dy}{dx} - 4 -2\frac{dy}{dx} = 0\\
       &\frac{dy}{dx} = \frac{2-x}{y-1}
    \end{align*}
    \textbf{B)} To find the coodrinates of those points on the curve where
    the tangent line is vertical we just need to find places where the derivative
    doesn't exist (the function is not differentiable) therefore we need to find the "limit" of the derivative so that
    it will be equal to $\pm \infty$.
    \begin{align*}
        lim_{x\to a} (\frac{2-x}{y-1}) = \pm \infty \\
    \end{align*}
    We can see that the derivative doesn't exist at y=1. If we substitute
    y=1 into the curve formula we find two solutions $\Rightarrow$ (0, 1) and (4, 1)

    \item \textbf{A)} In this part we'll once again use the implicit differentitation.
    \begin{align*}
        k'(x) = f'(g(x))\cdot g'(x)\\
        k'(-1) = f'(1) \cdot 1 = -3
    \end{align*}
    \textbf{B)} With solving the B part will help us the inverse derivave rules and then we'll hust substitute the slope to the point-slope formula.
    \begin{align*}
        &j'(1) = \frac{1}{f'(j(1))} = \frac{1}{2}\\
        &\text{Now we'll substitute into the point-slope formula at the point (1, -2)}\\
        &\Rightarrow y+2 = \frac{1}{2}(x-1)
    \end{align*}
\end{enumerate}
\end{document}